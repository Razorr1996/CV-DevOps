%-------------------------
% Rezume, a latex resume template for developers
% Author : Nanu Panchamurthy
% Based off of: https://github.com/sb2nov/resume
% License : MIT

% Hope this resume template helps you land an awesome job. If you found this helpful, please consider starring the github repo here, .
%-------------------------

%------------PACKAGES----------------
\documentclass[a4paper,11pt]{article}

\usepackage{verbatim} % reimplements the "verbatim" and "verbatim*" environments

\usepackage{titlesec} % provides an interface to sectioning commands i.e. custom elements

\usepackage{color} % provides both foreground and background color management

\usepackage{enumitem} % provides control over enumerate, itemize and description

\usepackage{fancyhdr} % provides extensive facilities for constructing headers, footers and also controlling their use

\usepackage{tabularx} % defines an environment tabularx, extension of "tabular" with an extra designator x, paragraph like column whose width automatically expands to fill the width of the environment

\usepackage{latexsym} % provides mathematical symbols

\usepackage{marvosym} % provides martin vogel's symbol font which contains various symbols

\usepackage[empty]{fullpage} % sets margins to one inch and removes headers, footers etc..

\usepackage[hidelinks]{hyperref} % removes color and shadow of hyperlinks

\usepackage[normalem]{ulem} % provides "\ul" (uline) command which will break at line breaks

\usepackage[english]{babel} % provides culturally determined typographical rules for wide range of languages

\setlength {\marginparwidth}{2cm}
\usepackage{todonotes}
%-----------------------------------------

\input glyphtounicode % converts glyph names to unicode
\pdfgentounicode=1 % ensures pdfs generated are ats readable

%----------FONT OPTIONS-------------------
\usepackage[default]{sourcesanspro} % uses the font source sans pro
\urlstyle{same} % changes url font from default urlfont to font being used by the document
%-----------------------------------------


%----------MARGIN OPTIONS-----------------
\pagestyle{fancy} % set page style to one configured by fancyhdr
\fancyhf{} % clear all header and footer fields

\renewcommand{\headrulewidth}{0in} % sets thickness of linerule under header to zero
\renewcommand{\footrulewidth}{0in} % sets thickness of linerule over footer to zero

\setlength{\tabcolsep}{0in} % sets thickness of column separator in tables to zero

% origin of the document is one inch from the top and from and the left
% oddsidemargin and evensidemargin both refer to the left margin
% right margin is indirectly set using oddsidemargin
\addtolength{\oddsidemargin}{-0.5in}
\addtolength{\topmargin}{-0.5in}

\addtolength{\textwidth}{1.0in} % sets width of text area in the page to one inch
\addtolength{\textheight}{1.0in} % sets height of text area in the page to one inch

\raggedbottom{} % makes all pages the height of current page, no extra vertical space added
\raggedright{} % makes all pages the width of current page, no extra horizontal space added
%------------------------------------------


%--------SECTIONING COMMANDS---------
% \titleformat{<command>}
%   [<shape>]{<format>}{<label>}{<sep>}
%     {<before-code>}[<after-code>]

% command is the sectioning command to be redefined
% shape is the style of the font; scshape stands for small caps style
% format is the format to be applied to whole title- label and text; absent here
% label defines the label
% sep is the horizontal separation between label and title body
% before-code is the code to be executed before
% after-code is the code to be executed after

\titleformat{\section}
  {\scshape\large}{}
    {0em}{\color{blue}}[\color{black}\titlerule\vspace{0pt}]
%-------------------------------------


%--------REDEFINITIONS----------------
% redefines the style of the bullet point
\renewcommand\labelitemii{$\vcenter{\hbox{\tiny$\bullet$}}$}

% redefines the underline depth to 2pt
\renewcommand{\ULdepth}{2pt}
%-------------------------------------


%--------CUSTOM COMMANDS--------------
%\vspace{} defines a vertical space of given size, modifying this in custom commands can help stretch or shrink resume to remove or add content

% resumeItem renders a bullet point
\newcommand{\resumeItem}[1]{
  \item\small{#1}
}

% commands to start and end itemization of resumeItem, rightmargin set to 0.11in to avoid the overflow of resumetItem beyond whatever resumeItemHeading is being used
\newcommand{\resumeItemListStart}{\begin{itemize}[rightmargin=0.11in]}
\newcommand{\resumeItemListEnd}{\end{itemize}}

% resumeSectionType renders a bolded type to be used under a section, used as skill type here, middle element is used to keep ":"s in the same vertical line
\newcommand{\resumeSectionType}[3]{
  \item\begin{tabular*}{0.96\textwidth}[t]{
    p{0.15\linewidth}p{0.02\linewidth}p{0.81\linewidth}
  }
    \textbf{#1} & #2 & #3
  \end{tabular*}\vspace{-2pt}
}

% resumeTrioHeading renders three elements in three columns with second element being italicized and first element bolded, can be used for projects with three elements
\newcommand{\resumeTrioHeading}[3]{
  \item\small{
    \begin{tabular*}{0.96\textwidth}[t]{
      l@{\extracolsep{\fill}}c@{\extracolsep{\fill}}r
    }
      \textbf{#1} & \textit{#2} & #3
    \end{tabular*}
  }
}

% resumeQuadHeading renders four elements in a two columns with the second row being italicized and first element of first row bolded, can be used for experience and projects with four elements
\newcommand{\resumeQuadHeading}[4]{
  \item
  \begin{tabular*}{0.96\textwidth}[t]{l@{\extracolsep{\fill}}r}
    \textbf{#1} & #2 \\
    \textit{\small#3} & \textit{\small #4} \\
  \end{tabular*}
}

% resumeQuadHeadingChild renders the second row of resumeQuadHeading, can be used for experience if different roles in the same company need to added
\newcommand{\resumeQuadHeadingChild}[2]{
  \item
  \begin{tabular*}{0.96\textwidth}[t]{l@{\extracolsep{\fill}}r}
    \textbf{\small#1} & {\small#2} \\
  \end{tabular*}
}

% commands to start and end itemization of resumeQuadHeading, lefmargin for left indent of 0.15in for resumeItems
\newcommand{\resumeHeadingListStart}{
  \begin{itemize}[leftmargin=0.15in, label={}]
}
\newcommand{\resumeHeadingListEnd}{\end{itemize}}
%-------------------------------------------


%__________________RESUME____________________
% You can rearrange sections in any order you may prefer
\begin{document}
\setlength{\footskip}{5pt}

%-----------CONTACT DETAILS------------------
% Make sure all the details are correct, you can add more links in the first row of second column if needed

\begin{tabular*}{\textwidth}{l@{\extracolsep{\fill}}r}
  \textbf{\Huge Artem Basalaev \vspace{2pt}} &
  {Location: Budva, Montenegro} \\
  LinkedIn: \href{https://linkedin.com/in/artem-basalaev}{\uline{artem-basalaev}} $|$
  GitLab: \href{https://gitlab.com/Razorr1996}{\uline{@Razorr1996}} $|$
  GitHub: \href{https://github.com/Razorr1996}{\uline{@Razorr1996}} \\
  Email: \href{mailto:basa62.1996@gmail.com}{\uline{basa62.1996@gmail.com}} $|$
  Mobile: \href{tel:+79819547080}{\uline{+7(981)954-70-80}} / \href{tel:+38269532916}{\uline{+382 69 532916}} \\ % row = 2, col = 2
\end{tabular*}
%--------------------------------------------


%-----------SUMMARY--------------------------
% Keep this short, simple and straigth to point

\section{DevOps Engineer}
\small{
I am a highly skilled DevOps engineer with over \textbf{6 years of experience} in \textbf{CI/CD, Cloud Providers, Web Servers, Network, and Linux}.
\todo{sum}
I have knowledge of popular frameworks such as \textbf{React, Angular, and Vue.js} and experience with REST APIs and MVC frameworks.

Languages: \textbf{Russian} (Native), \textbf{English} (Upper Intermediate).
}
%--------------------------------------------


%--------------SKILLS------------------------
% Add or remove resumeSectionTypes according to your needs

\section{Technical Skills}
\todo{skills}
  \resumeHeadingListStart{}
    \resumeSectionType{Languages}{:}{Python, Bash, Rust, Java}
    \resumeSectionType{Automatization}{:}{GitLab CI, Terraform, TeamCity, Ansible}
    \resumeSectionType{Dev Tools}{:}{Docker, Git, IntelliJ IDEA}
    \resumeSectionType{Cloud}{:}{AWS, GCP}
    \resumeSectionType{Databases}{:}{PostgreSQL}
  \resumeHeadingListEnd{}
%--------------------------------------------


%-----------EXPERIENCE-----------------------
% Distill all your talking points to small bullet points which follow the pattern "challenge-action-result" for maximum efficiency. Try to quantify (use numbers) your points whenver possible, highlist words of importance

\section{Experience}
\resumeHeadingListStart{}
  \resumeQuadHeading{DevOps Lead}{Aug 2021 -- Present}
  {\href{https://www.linkedin.com/company/dualbootpartners/}{\uline{Dualboot Partners}}}{Remote}
  \todo{w3}
    % \resumeItemListStart{}
    %   \resumeItem{Designed and developed dynamic and responsive websites using \textbf{HTML, CSS, JavaScript, and PHP}}
    %   \resumeItem{Worked with \textbf{REST APIs} to retrieve and display data from databases}
    %   \resumeItem{Improved \textbf{website performance} and speed through optimization techniques by \textbf{55\%}}
    % \resumeItemListEnd{}

  \resumeQuadHeading{System Administrator}{Jan 2019 -- Aug 2021}
  {\href{https://www.linkedin.com/company/solanteq/}{\uline{Solanteq}}}{Saint Petersburg, Russia}
  \todo{w2}
    % \resumeItemListStart{}
    %   \resumeItem{Worked with \textbf{MVC frameworks} to develop robust and scalable backends}
    %   \resumeItem{Troubleshot and \textbf{fixed bugs} and issues in the backend to ensure \textbf{smooth operation} of the applications}
    % \resumeItemListEnd{}

  \resumeQuadHeading{Developer/System Administrator}{May 2017 -- Dec 2018}{\href{https://sim-lab.ru}{\uline{Computer Modelling Laboratory}}}{Saint Petersburg, Russia}
    \resumeItemListStart{}
      \resumeItem{Configured CI pipelines for various products (Python, C\#, Unity) using Docker.}
      \resumeItem{Managed web server, proxy, and load balancer configurations with Nginx.}
      \resumeItem{Administered CI/CD services: GitLab and TeamCity.}
      \resumeItem{Maintained Linux and Windows Server-based servers.}
      \resumeItem{Orchestrated virtual machines on KVM and Hyper-V.}
    \resumeItemListEnd{}
\resumeHeadingListEnd{}
%---------------------------------------------


%-----------EDUCATION-------------------------
% Mention your CGPA, if its good, in the first row of second column

\section{Education}
  \resumeHeadingListStart{}
    \resumeQuadHeading{ITMO University}{Sep 2018 -- Aug 2020}
    {Master of Science in Computer Science}{Saint Petersburg, Russia}
    \resumeQuadHeading{ITMO University}{Sep 2014 -- Aug 2018}
    {Bachelor of Science in Computer Science}{Saint Petersburg, Russia}
  \resumeHeadingListEnd{}
%---------------------------------------------


%-----------PROJECTS--------------------------
% Use resumeQuadHeading if four elements are feasible (ex: demo video link), else use resumeTrioHeading. Keep the bullet points simple and concise and try to cover wide variety of skills you have used to build these projects

% \section{Projects}
% \todo{prj}
%   \resumeHeadingListStart{}
%     \resumeTrioHeading{\href{https://project1.com}{\uline{Project 1}}}{React.js, Redux, PHP, MySQL Git}{\href{https://proect1.com/source-code/}{\uline{Source Code}}}
%       \resumeItemListStart{}
%         \resumeItem{Designed and developed a clean and modern website using \textbf{HTML, CSS, and JavaScript}}
%         \resumeItem{Optimized website for \textbf{speed and user experience}}
%         \resumeItem{Utilized \textbf{responsive design} to ensure compatibility across all devices}
%         \resumeItem{Deployed on GitHub pages via GitHub Actions}
%       \resumeItemListEnd{}

%       \resumeTrioHeading{Project 2}{Node.js, Express, JavaScript, Git}{\href{https:project2.com/source-code}{\uline{Source Code}}}
%       \resumeItemListStart{}
%         \resumeItem{A \textbf{CRUD application} exposed using a RESTful API made with Node.js}
%         \resumeItem{Exposed POST, GET, PATCH and DELETE HTTP methods using \textbf{Express}}
%       \resumeItemListEnd{}
%   \resumeHeadingListEnd{}
%--------------------------------------------


%----------------OTHERS----------------------
% You can add your acheivements, accolades, certifications etc. here.

\section{Certifications}
  \resumeItemListStart{}
    \resumeItem{\href{https://www.credly.com/badges/968bad65-6464-49fb-b836-c53a0dd30daf/linked_in_profile}{\uline{AWS Certified Cloud Practitioner}}}
    \resumeItem{\href{https://coursera.org/verify/ND55XKAHNRU4}{\uline{"Getting Started with Google Kubernetes Engine" by Google Cloud Training}}}
    \resumeItem{\href{https://stepik.org/cert/151752?lang=en}{\uline{"Algorithms: theory and practice. Data structures" by Computer Science Center}}}
    \resumeItem{\href{https://stepik.org/cert/39100?lang=en}{\uline{"Algorithms: theory and practice. Methods" by Computer Science Center}}}
  \resumeItemListEnd{}
%--------------------------------------------

\end{document}
