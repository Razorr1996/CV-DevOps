%-------------------------
% Rezume, a latex resume template for developers
% Author : Nanu Panchamurthy
% Based off of: https://github.com/sb2nov/resume
% License : MIT

% Hope this resume template helps you land an awesome job. If you found this helpful, please consider starring the github repo here, .
%-------------------------

%------------PACKAGES----------------
\documentclass[a4paper,10pt]{article}

\usepackage{verbatim} % reimplements the "verbatim" and "verbatim*" environments

\usepackage{titlesec} % provides an interface to sectioning commands i.e. custom elements

\usepackage{color} % provides both foreground and background color management

\usepackage{enumitem} % provides control over enumerate, itemize and description

\usepackage{fancyhdr} % provides extensive facilities for constructing headers, footers and also controlling their use

\usepackage{tabularx} % defines an environment tabularx, extension of "tabular" with an extra designator x, paragraph like column whose width automatically expands to fill the width of the environment

\usepackage{latexsym} % provides mathematical symbols

\usepackage{marvosym} % provides martin vogel's symbol font which contains various symbols

\usepackage[empty]{fullpage} % sets margins to one inch and removes headers, footers etc..

\usepackage[hidelinks]{hyperref} % removes color and shadow of hyperlinks

\usepackage[normalem]{ulem} % provides "\ul" (uline) command which will break at line breaks

\usepackage[english]{babel} % provides culturally determined typographical rules for wide range of languages

\setlength {\marginparwidth}{2cm}
\usepackage{todonotes}
%-----------------------------------------

\input glyphtounicode % converts glyph names to unicode
\pdfgentounicode=1 % ensures pdfs generated are ats readable

%----------FONT OPTIONS-------------------
\usepackage[default]{sourcesanspro} % uses the font source sans pro
\urlstyle{same} % changes url font from default urlfont to font being used by the document
%-----------------------------------------


%----------MARGIN OPTIONS-----------------
\pagestyle{fancy} % set page style to one configured by fancyhdr
\fancyhf{} % clear all header and footer fields

\renewcommand{\headrulewidth}{0in} % sets thickness of linerule under header to zero
\renewcommand{\footrulewidth}{0in} % sets thickness of linerule over footer to zero

\setlength{\tabcolsep}{0in} % sets thickness of column separator in tables to zero

% origin of the document is one inch from the top and from and the left
% oddsidemargin and evensidemargin both refer to the left margin
% right margin is indirectly set using oddsidemargin
\addtolength{\oddsidemargin}{-0.5in}
\addtolength{\topmargin}{-0.5in}

\addtolength{\textwidth}{1.0in} % sets width of text area in the page to one inch
\addtolength{\textheight}{1.0in} % sets height of text area in the page to one inch

\raggedbottom{} % makes all pages the height of current page, no extra vertical space added
\raggedright{} % makes all pages the width of current page, no extra horizontal space added
%------------------------------------------


%--------SECTIONING COMMANDS---------
% \titleformat{<command>}
%   [<shape>]{<format>}{<label>}{<sep>}
%     {<before-code>}[<after-code>]

% command is the sectioning command to be redefined
% shape is the style of the font; scshape stands for small caps style
% format is the format to be applied to whole title- label and text; absent here
% label defines the label
% sep is the horizontal separation between label and title body
% before-code is the code to be executed before
% after-code is the code to be executed after

\titleformat{\section}
  {\scshape\large}{}
    {0em}{\color{blue}}[\color{black}\titlerule\vspace{0pt}]
%-------------------------------------


%--------REDEFINITIONS----------------
% redefines the style of the bullet point
\renewcommand\labelitemii{$\vcenter{\hbox{\tiny$\bullet$}}$}

% redefines the underline depth to 2pt
\renewcommand{\ULdepth}{2pt}
%-------------------------------------


%--------CUSTOM COMMANDS--------------
%\vspace{} defines a vertical space of given size, modifying this in custom commands can help stretch or shrink resume to remove or add content

% resumeItem renders a bullet point
\newcommand{\resumeItem}[1]{
  \item\small{#1}
}

% commands to start and end itemization of resumeItem, rightmargin set to 0.11in to avoid the overflow of resumetItem beyond whatever resumeItemHeading is being used
\newcommand{\resumeItemListStart}{\begin{itemize}[rightmargin=0.11in]}
\newcommand{\resumeItemListEnd}{\end{itemize}}

% resumeSectionType renders a bolded type to be used under a section, used as skill type here, middle element is used to keep ":"s in the same vertical line
\newcommand{\resumeSectionType}[3]{
  \item\begin{tabular*}{0.96\textwidth}[t]{
    p{0.15\linewidth}p{0.02\linewidth}p{0.81\linewidth}
  }
    \textbf{#1} & #2 & #3
  \end{tabular*}\vspace{-2pt}
}

% resumeTrioHeading renders three elements in three columns with second element being italicized and first element bolded, can be used for projects with three elements
\newcommand{\resumeTrioHeading}[3]{
  \item\small{
    \begin{tabular*}{0.96\textwidth}[t]{
      l@{\extracolsep{\fill}}c@{\extracolsep{\fill}}r
    }
      \textbf{#1} & \textit{#2} & #3
    \end{tabular*}
  }
}

% resumeQuadHeading renders four elements in a two columns with the second row being italicized and first element of first row bolded, can be used for experience and projects with four elements
\newcommand{\resumeQuadHeading}[4]{
  \item
  \begin{tabular*}{0.96\textwidth}[t]{l@{\extracolsep{\fill}}r}
    \textbf{#1} & #2 \\
    \textit{\small#3} & \textit{\small #4} \\
  \end{tabular*}
}

% resumeQuadHeadingChild renders the second row of resumeQuadHeading, can be used for experience if different roles in the same company need to added
\newcommand{\resumeQuadHeadingChild}[2]{
  \item
  \begin{tabular*}{0.96\textwidth}[t]{l@{\extracolsep{\fill}}r}
    \textbf{\small#1} & {\small#2} \\
  \end{tabular*}
}

% commands to start and end itemization of resumeQuadHeading, lefmargin for left indent of 0.15in for resumeItems
\newcommand{\resumeHeadingListStart}{
  \begin{itemize}[leftmargin=0.15in, label={}]
}
\newcommand{\resumeHeadingListEnd}{\end{itemize}}

\newcommand{\hrefUline}[2]{
  \href{#1}{\uline{#2}}
}
%-------------------------------------------


%__________________RESUME____________________
% You can rearrange sections in any order you may prefer
\begin{document}
\setlength{\footskip}{5pt}

%-----------CONTACT DETAILS------------------
% Make sure all the details are correct, you can add more links in the first row of second column if needed

\begin{tabular*}{\textwidth}{l@{\extracolsep{\fill}}r}
  \textbf{\Huge Artem Basalaev} &
  {Location: Belgrade, Serbia} \\
  LinkedIn:\hrefUline{https://linkedin.com/in/artem-basalaev}{in/artem-basalaev} $|$
  GitHub:\hrefUline{https://github.com/Razorr1996}{@Razorr1996} $|$
  GitLab:\hrefUline{https://gitlab.com/Razorr1996}{@Razorr1996} \\
  Email:\hrefUline{mailto:basa62.1996@gmail.com}{basa62.1996@gmail.com} $|$
  Mobile:\hrefUline{tel:+79819547080}{+7 981 954-70-80}/\hrefUline{tel:+381621606662}{+381 62 160-666-2} $|$
  Telegram:\hrefUline{https://t.me/basa62}{@basa62} \\
\end{tabular*}
%--------------------------------------------


%-----------SUMMARY--------------------------
% Keep this short, simple and straigth to point

\section{DevOps Engineer}
\small{
I am a highly skilled DevOps engineer and \textbf{Certified Kubernetes Administrator} with over \textbf{7 years of experience} in \textbf{Cloud Providers, Container Orchestration, CI/CD, Linux, Web Servers, and Networks}.

Languages: \textbf{Russian} (Native), \textbf{English} (Upper Intermediate), \textbf{Serbian} (Elementary).
}
%--------------------------------------------


%--------------SKILLS------------------------
% Add or remove resumeSectionTypes according to your needs

\section{Technical Skills}
  \resumeHeadingListStart{}
    \resumeSectionType{Containers}{:}{Kubernetes, EKS, ECS, Docker}
    \resumeSectionType{Automation}{:}{Terraform, GitLab CI, FluxCD, ArgoCD, Ansible, TeamCity}
    \resumeSectionType{Cloud}{:}{AWS, GCP}
    \resumeSectionType{Monitoring}{:}{Prometheus, Grafana, Loki, CloudWatch}
    \resumeSectionType{Languages}{:}{Python, Bash, Rust, Java}
    \resumeSectionType{Dev Tools}{:}{Git, Docker Compose, IntelliJ IDEA}
    \resumeSectionType{Databases}{:}{PostgreSQL}
  \resumeHeadingListEnd{}
%--------------------------------------------


%-----------EXPERIENCE-----------------------
% Distill all your talking points to small bullet points which follow the pattern "challenge-action-result" for maximum efficiency. Try to quantify (use numbers) your points whenver possible, highlist words of importance

\section{Experience}
\resumeHeadingListStart{}
  \resumeQuadHeading{Senior DevOps Engineer}{Apr 2024 -- Present}
  {\hrefUline{https://www.linkedin.com/company/naviteq-solutions/}{Naviteq}}{Remote}

  \resumeQuadHeading{Senior DevOps Engineer}{Aug 2021 -- Apr 2024}
  {\hrefUline{https://www.linkedin.com/company/dualbootpartners/}{Dualboot Partners}}{Remote}
  \resumeItemListStart{}
    \resumeItem{Developed cloud infrastructure for various projects (AWS, Kubernetes, Terraform, Docker, GitLab CI, Ruby/Python/Node.js, etc.).}
    \resumeItem{Integrated solution for running dynamic environments per feature branch using Uffizzi, Docker, GitLab CI, AWS, and Traefik.}
    \resumeItem{Migrated applications to containers and Cloud Providers.}
    \resumeItem{Manage work for infrastructure across multiple company projects.}
    \resumeItem{Have experience as a Team Lead on the Python Django project.}
  \resumeItemListEnd{}

  \resumeQuadHeading{DevOps Engineer}{Jan 2019 -- Aug 2021}
  {\hrefUline{https://www.linkedin.com/company/solanteq/}{Solanteq}}{Saint Petersburg, Russia}
    \resumeItemListStart{}
      \resumeItem{Designed a system for dynamic environments powered by AWS EC2+RDS, AWS SDK (boto3), and CI/CD.}
      \resumeItem{Created Ansible Playbooks to distribute JVM+Docker applications.}
      \resumeItem{Configured CI/CD templates for JVM applications.}
      \resumeItem{Maintained company's infrastructure (Cloud \& On-Premise): Linux/Windows-servers, GitLab, YouTrack, TeamCity, E-Mail, laptop configuration, office network, etc.}
    \resumeItemListEnd{}

  \resumeQuadHeading{Developer/System Administrator}{May 2017 -- Dec 2018}
  {\hrefUline{https://sim-lab.ru}{Computer Modelling Laboratory}}{Saint Petersburg, Russia}
    \resumeItemListStart{}
      \resumeItem{Configured CI pipelines for various products (Python, C\#, Unity).}
      \resumeItem{Managed web server, proxy, and load balancer configurations with Nginx.}
      \resumeItem{Maintained Linux/Windows servers: virtual machines (KVM, Hyper-V), CI/CD services (GitLab, TeamCity).}
    \resumeItemListEnd{}
\resumeHeadingListEnd{}
%---------------------------------------------


%-----------EDUCATION-------------------------
% Mention your CGPA, if its good, in the first row of second column

\section{Education}
  \resumeHeadingListStart{}
    \resumeQuadHeading{ITMO University}{Sep 2018 -- Aug 2020}
    {Master of Science in Computer Science}{Saint Petersburg, Russia}
      \resumeItemListStart{}
        \resumeItem{Major: Software Engineering.}
      \resumeItemListEnd{}

    \resumeQuadHeading{ITMO University}{Sep 2014 -- Aug 2018}
    {Bachelor of Science in Computer Science}{Saint Petersburg, Russia}
      \resumeItemListStart{}
        \resumeItem{Majors: System Software, Database Management Systems.}
      \resumeItemListEnd{}
  \resumeHeadingListEnd{}
%---------------------------------------------


%-----------PROJECTS--------------------------
% Use resumeQuadHeading if four elements are feasible (ex: demo video link), else use resumeTrioHeading. Keep the bullet points simple and concise and try to cover wide variety of skills you have used to build these projects

% \section{Projects}
%   \resumeHeadingListStart{}
%     \resumeTrioHeading{\hrefUline{https://project1.com}{Project 1}}{React.js, Redux, PHP, MySQL Git}{\hrefUline{https://proect1.com/source-code/}{Source Code}}
%       \resumeItemListStart{}
%         \resumeItem{Designed and developed a clean and modern website using \textbf{HTML, CSS, and JavaScript}}
%         \resumeItem{Optimized website for \textbf{speed and user experience}}
%         \resumeItem{Utilized \textbf{responsive design} to ensure compatibility across all devices}
%         \resumeItem{Deployed on GitHub pages via GitHub Actions}
%       \resumeItemListEnd{}

%       \resumeTrioHeading{Project 2}{Node.js, Express, JavaScript, Git}{\hrefUline{https:project2.com/source-code}{Source Code}}
%       \resumeItemListStart{}
%         \resumeItem{A \textbf{CRUD application} exposed using a RESTful API made with Node.js}
%         \resumeItem{Exposed POST, GET, PATCH and DELETE HTTP methods using \textbf{Express}}
%       \resumeItemListEnd{}
%   \resumeHeadingListEnd{}
%--------------------------------------------


%----------------OTHERS----------------------
% You can add your acheivements, accolades, certifications etc. here.

\section{Certifications}
  \resumeItemListStart{}
    \resumeItem{\hrefUline{https://www.credly.com/badges/24659e03-b038-41e8-9a85-9c10fbe525d5}{CKA: Certified Kubernetes Administrator}}
    \resumeItem{\hrefUline{https://www.credly.com/badges/968bad65-6464-49fb-b836-c53a0dd30daf}{AWS Certified Cloud Practitioner}}
  \resumeItemListEnd{}
%--------------------------------------------

\end{document}
